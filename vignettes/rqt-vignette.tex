\documentclass{article}
\usepackage{fullpage}
\usepackage{hyperref}
\usepackage{fixltx2e}

%\VignetteIndexEntry{Using RQT}

\title{Using RQT, an R package for gene-level meta-analysis}
\author{Ilya Y. Zhbannikov}
\date{\today}


\usepackage{Sweave}
\begin{document}
\Sconcordance{concordance:rqt-vignette.tex:rqt-vignette.Rnw:%
1 12 1 1 0 36 1 1 2 1 0 1 1 3 0 1 2 20 1 1 2 1 0 1 2 1 0 5 1 81 0 1 2 2 %
1 1 2 1 0 1 2 1 0 5 1 81 0 1 2 3 1 1 2 1 0 1 2 1 0 5 1 81 0 1 2 4 1 1 2 %
1 0 1 2 1 0 4 1 6 0 1 1 81 0 1 2 11 1 1 2 1 0 1 2 1 0 2 1 1 2 1 0 3 1 %
87 0 1 2 3 1 1 2 1 0 8 1 16 0 1 2 2 1 1 2 24 0 1 2 28 1}

\maketitle

\section{Overview}

Despite the recent advances of modern GWAS methods, 
it is still remains an important problem of addressing 
calculation an effect size and corresponding p-value 
for the whole gene rather than for single variant. 
We developed an R-package rqt, which offers gene-level GWAS meta-analysis. 
The package can be easily included into bioinformatics pipeline 
or used as stand-alone. 
Contact: ilya.zhbannikov@duke.edu for questions of 
usage the \texttt{rqt} or any other issues.

Below we provide several examples that show GWAS 
meta-analysis on gene-level layer.

\subsection{Methods in brief}

The workflow of gene-level meta analysis consists of the following steps: 
(i) reducing the number of predictors, thereby alleviating 
correlation problem in variants (accounting for LD); 
(ii) then the regression mod-el is fitted on the reduced dataset 
to obtain corresponding regression coefficient ("effect sizes"); 
(iii) these coefficients are then to be pooled into a total index 
representing a total gene-level effect size and corresponding 
statistics is calculated. P- and q- values are then calculated 
using this statistics from asymptotic approximation or permutation 
procedure; (iv) the final step is combining gene-level p-values 
calculated from each study with Fisher's combined probability method.

\section{Installation of \emph{rqt} package}
In order to install the \emph{rqt} package, the user must first install R (\url{http://www.r-project.org}). After that, \emph{rqt} can be installed with:

\begin{Schunk}
\begin{Sinput}
> source("http://bioconductor.org/biocLite.R")
> biocLite(c("rqt"))
\end{Sinput}
\end{Schunk}


\section{Data description}
\subsection{Single dataset}
In \texttt{rqt} requires the following datasets: 
(i) \texttt{phenotype} (a \texttt{N} by 1) matrix (i.e. a vector); 
and (ii) \texttt{genotype} - an object of class 
\texttt{SummarizedExperiment} containing one assay:
(a \texttt{N} by \texttt{M}) matrix, where 
\texttt{N} - is the total 
number of individuals in the study and \texttt{M} is the total number 
of genetic variants. Optionally, \texttt{rqt} can accept covariates, 
in form of \texttt{N} by \texttt{K} matrix, where \texttt{K} 
is the total number of 
covariates used in the study. Phenotype can be dichotomous 
(0/1, where 1 indicates control and 0 case).

\subsection{Meta-analysis}
In meta-analysis, \texttt{rqt} requires a list of \texttt{M} 
(\texttt{M} - number 
of datasets used in meta-analysis) and optionally it accepts
covariates in form described above.

\section{Examples}

\subsection{Gene-level analysis on a single dataset}
\subsection{Dichotomous phenotype}

\begin{Schunk}
\begin{Sinput}
> library(rqt)
> data <- data.matrix(read.table(system.file("extdata/test.bin1.dat",
+                                            package="rqt"), header=TRUE))
> pheno <- data[,1]
> geno <- data[, 2:dim(data)[2]]
> colnames(geno) <- paste(seq(1, dim(geno)[2]))
> geno.obj <- SummarizedExperiment(geno)
> obj <- rqtClass(phenotype=pheno, genotype=geno.obj)
> res <- rQTest(obj, method="pca", out.type = "D")
> print(res)
\end{Sinput}
\begin{Soutput}
Phenotype:
[1] 1 1 1 1 1 1
...

Genotype:
     1 2 3 4 5 6 7 8 9 10 11 12 13 14 15 16 17 18 19 20 21 22 23 24 25 26 27 28
[1,] 0 0 0 1 0 2 1 0 0  0  2  1  2  0  0  2  1  1  0  0  1  0  0  0  0  2  1  0
[2,] 1 0 1 0 0 1 0 0 0  0  0  0  1  0  0  0  2  1  2  1  0  1  0  0  0  2  1  1
[3,] 0 0 0 0 1 0 0 1 0  1  1  1  0  0  0  1  1  0  0  1  1  1  0  0  0  1  0  0
[4,] 0 0 1 0 1 0 0 1 1  0  0  0  1  0  0  0  1  0  0  2  0  1  0  0  0  1  0  2
[5,] 0 0 1 1 1 1 1 1 0  1  0  1  0  0  0  0  0  0  0  0  1  0  0  1  0  1  2  0
[6,] 0 0 1 1 1 0 0 1 0  1  1  0  1  0  0  2  0  0  1  0  1  1  0  0  0  2  0  0
     29 30 31 32 33 34 35 36 37 38 39 40 41 42 43 44 45 46 47 48 49 50 51 52 53
[1,]  1  0  0  0  0  0  0  1  1  0  0  1  0  0  1  0  0  0  0  1  1  0  0  0  1
[2,]  2  0  0  0  1  2  1  2  0  1  1  0  1  0  0  1  0  0  0  2  0  1  0  1  0
[3,]  0  0  0  0  0  1  1  1  2  2  0  0  0  1  0  1  0  2  1  1  1  0  0  0  1
[4,]  0  0  0  1  2  2  0  1  1  1  1  0  0  0  1  1  0  0  1  1  2  0  1  0  2
[5,]  0  0  0  0  1  1  0  0  1  1  0  0  2  2  0  0  0  1  2  1  0  0  0  0  1
[6,]  0  0  0  0  1  0  1  1  0  0  2  0  1  1  2  0  0  0  1  1  1  0  0  1  1
     54 55 56 57 58 59 60 61 62 63 64 65 66 67 68 69 70 71 72 73 74 75 76 77 78
[1,]  1  1  1  1  0  0  0  0  1  1  0  1  1  1  0  1  1  2  1  1  1  0  0  2  1
[2,]  2  1  0  1  0  0  0  1  0  1  2  2  1  1  0  1  0  1  0  0  0  0  1  1  2
[3,]  2  1  0  0  0  0  0  0  0  1  0  0  1  1  0  1  2  1  0  1  1  0  0  0  0
[4,]  1  1  0  0  0  0  2  1  0  1  0  0  2  1  0  0  1  1  1  0  0  0  0  0  1
[5,]  1  1  1  0  0  0  2  1  1  0  0  1  1  0  0  1  1  0  0  1  0  1  0  0  1
[6,]  0  1  1  0  0  0  0  0  0  0  1  2  1  1  0  0  0  0  0  1  1  1  1  1  1
     79 80 81 82 83 84 85 86 87 88 89 90 91 92 93 94 95 96 97 98 99 100
[1,]  1  1  0  1  1  0  1  1  0  0  0  1  0  1  1  0  0  0  1  0  1   1
[2,]  1  1  2  0  0  0  1  2  0  2  1  0  0  2  1  0  0  0  0  1  1   2
[3,]  1  1  0  0  0  0  2  0  0  2  0  0  0  0  1  1  0  0  1  2  0   0
[4,]  0  1  2  0  1  0  1  2  0  0  0  0  0  0  1  0  0  0  1  0  0   0
[5,]  0  2  0  1  0  0  2  0  0  0  0  1  0  0  1  0  0  1  1  2  1   1
[6,]  2  1  0  0  1  0  0  0  0  2  0  1  0  0  0  0  0  1  1  0  1   1
...

Covariates:
data frame with 0 columns and 0 rows


Results:

$Qstatistic
         Q1        Q2        Q3
1 0.9086994 0.9938212 0.9086994

$p.value
       p.Q1      p.Q2      p.Q3
1 0.3404597 0.9285409 0.5654533
\end{Soutput}
\end{Schunk}

\subsection{Continuous phenotype}

\begin{Schunk}
\begin{Sinput}
> library(rqt)
> data <- data.matrix(read.table(system.file("extdata/test.cont1.dat",
+                                            package="rqt"), header = TRUE))
> pheno <- data[,1]
> geno <- data[, 2:dim(data)[2]]
> colnames(geno) <- paste(seq(1, dim(geno)[2]))
> geno.obj <- SummarizedExperiment(geno)
> obj <- rqtClass(phenotype=pheno, genotype=geno.obj)
> res <- rQTest(obj, method="pca", out.type = "C")
> print(res)
\end{Sinput}
\begin{Soutput}
Phenotype:
[1] 3.422452 2.457394 2.708564 4.589394 5.461723 4.438707
...

Genotype:
     1 2 3 4 5 6 7 8 9 10 11 12 13 14 15 16 17 18 19 20 21 22 23 24 25 26 27 28
[1,] 0 0 0 1 0 2 1 0 0  0  2  1  2  0  0  2  1  1  0  0  1  0  0  0  0  2  1  0
[2,] 1 0 1 0 0 1 0 0 0  0  0  0  1  0  0  0  2  1  2  1  0  1  0  0  0  2  1  1
[3,] 0 0 0 0 1 0 0 1 0  1  1  1  0  0  0  1  1  0  0  1  1  1  0  0  0  1  0  0
[4,] 0 0 1 0 1 0 0 1 1  0  0  0  1  0  0  0  1  0  0  2  0  1  0  0  0  1  0  2
[5,] 0 0 1 1 1 1 1 1 0  1  0  1  0  0  0  0  0  0  0  0  1  0  0  1  0  1  2  0
[6,] 0 0 1 1 1 0 0 1 0  1  1  0  1  0  0  2  0  0  1  0  1  1  0  0  0  2  0  0
     29 30 31 32 33 34 35 36 37 38 39 40 41 42 43 44 45 46 47 48 49 50 51 52 53
[1,]  1  0  0  0  0  0  0  1  1  0  0  1  0  0  1  0  0  0  0  1  1  0  0  0  1
[2,]  2  0  0  0  1  2  1  2  0  1  1  0  1  0  0  1  0  0  0  2  0  1  0  1  0
[3,]  0  0  0  0  0  1  1  1  2  2  0  0  0  1  0  1  0  2  1  1  1  0  0  0  1
[4,]  0  0  0  1  2  2  0  1  1  1  1  0  0  0  1  1  0  0  1  1  2  0  1  0  2
[5,]  0  0  0  0  1  1  0  0  1  1  0  0  2  2  0  0  0  1  2  1  0  0  0  0  1
[6,]  0  0  0  0  1  0  1  1  0  0  2  0  1  1  2  0  0  0  1  1  1  0  0  1  1
     54 55 56 57 58 59 60 61 62 63 64 65 66 67 68 69 70 71 72 73 74 75 76 77 78
[1,]  1  1  1  1  0  0  0  0  1  1  0  1  1  1  0  1  1  2  1  1  1  0  0  2  1
[2,]  2  1  0  1  0  0  0  1  0  1  2  2  1  1  0  1  0  1  0  0  0  0  1  1  2
[3,]  2  1  0  0  0  0  0  0  0  1  0  0  1  1  0  1  2  1  0  1  1  0  0  0  0
[4,]  1  1  0  0  0  0  2  1  0  1  0  0  2  1  0  0  1  1  1  0  0  0  0  0  1
[5,]  1  1  1  0  0  0  2  1  1  0  0  1  1  0  0  1  1  0  0  1  0  1  0  0  1
[6,]  0  1  1  0  0  0  0  0  0  0  1  2  1  1  0  0  0  0  0  1  1  1  1  1  1
     79 80 81 82 83 84 85 86 87 88 89 90 91 92 93 94 95 96 97 98 99 100
[1,]  1  1  0  1  1  0  1  1  0  0  0  1  0  1  1  0  0  0  1  0  1   1
[2,]  1  1  2  0  0  0  1  2  0  2  1  0  0  2  1  0  0  0  0  1  1   2
[3,]  1  1  0  0  0  0  2  0  0  2  0  0  0  0  1  1  0  0  1  2  0   0
[4,]  0  1  2  0  1  0  1  2  0  0  0  0  0  0  1  0  0  0  1  0  0   0
[5,]  0  2  0  1  0  0  2  0  0  0  0  1  0  0  1  0  0  1  1  2  1   1
[6,]  2  1  0  0  1  0  0  0  0  2  0  1  0  0  0  0  0  1  1  0  1   1
...

Covariates:
data frame with 0 columns and 0 rows


Results:

$Qstatistic
         Q1       Q2       Q3
1 0.2846585 2.389537 2.219594

$p.value
      p.Q1      p.Q2      p.Q3
1 0.593664 0.7022459 0.2561996
\end{Soutput}
\end{Schunk}

\subsection{Preprocessing with Partial Least Square regression (PLS)}

This method is used for continous outcome, i.e. out.type = "C".
\begin{Schunk}
\begin{Sinput}
> library(rqt)
> data <- data.matrix(read.table(system.file("extdata/test.cont1.dat",
+                                            package="rqt"), header = TRUE))
> pheno <- data[,1]
> geno <- data[, 2:dim(data)[2]]
> colnames(geno) <- paste(seq(1, dim(geno)[2]))
> geno.obj <- SummarizedExperiment(geno)
> obj <- rqtClass(phenotype=pheno, genotype=geno.obj)
> res <- rQTest(obj, method="pls", out.type = "C")
> print(res)
\end{Sinput}
\begin{Soutput}
Phenotype:
[1] 3.422452 2.457394 2.708564 4.589394 5.461723 4.438707
...

Genotype:
     1 2 3 4 5 6 7 8 9 10 11 12 13 14 15 16 17 18 19 20 21 22 23 24 25 26 27 28
[1,] 0 0 0 1 0 2 1 0 0  0  2  1  2  0  0  2  1  1  0  0  1  0  0  0  0  2  1  0
[2,] 1 0 1 0 0 1 0 0 0  0  0  0  1  0  0  0  2  1  2  1  0  1  0  0  0  2  1  1
[3,] 0 0 0 0 1 0 0 1 0  1  1  1  0  0  0  1  1  0  0  1  1  1  0  0  0  1  0  0
[4,] 0 0 1 0 1 0 0 1 1  0  0  0  1  0  0  0  1  0  0  2  0  1  0  0  0  1  0  2
[5,] 0 0 1 1 1 1 1 1 0  1  0  1  0  0  0  0  0  0  0  0  1  0  0  1  0  1  2  0
[6,] 0 0 1 1 1 0 0 1 0  1  1  0  1  0  0  2  0  0  1  0  1  1  0  0  0  2  0  0
     29 30 31 32 33 34 35 36 37 38 39 40 41 42 43 44 45 46 47 48 49 50 51 52 53
[1,]  1  0  0  0  0  0  0  1  1  0  0  1  0  0  1  0  0  0  0  1  1  0  0  0  1
[2,]  2  0  0  0  1  2  1  2  0  1  1  0  1  0  0  1  0  0  0  2  0  1  0  1  0
[3,]  0  0  0  0  0  1  1  1  2  2  0  0  0  1  0  1  0  2  1  1  1  0  0  0  1
[4,]  0  0  0  1  2  2  0  1  1  1  1  0  0  0  1  1  0  0  1  1  2  0  1  0  2
[5,]  0  0  0  0  1  1  0  0  1  1  0  0  2  2  0  0  0  1  2  1  0  0  0  0  1
[6,]  0  0  0  0  1  0  1  1  0  0  2  0  1  1  2  0  0  0  1  1  1  0  0  1  1
     54 55 56 57 58 59 60 61 62 63 64 65 66 67 68 69 70 71 72 73 74 75 76 77 78
[1,]  1  1  1  1  0  0  0  0  1  1  0  1  1  1  0  1  1  2  1  1  1  0  0  2  1
[2,]  2  1  0  1  0  0  0  1  0  1  2  2  1  1  0  1  0  1  0  0  0  0  1  1  2
[3,]  2  1  0  0  0  0  0  0  0  1  0  0  1  1  0  1  2  1  0  1  1  0  0  0  0
[4,]  1  1  0  0  0  0  2  1  0  1  0  0  2  1  0  0  1  1  1  0  0  0  0  0  1
[5,]  1  1  1  0  0  0  2  1  1  0  0  1  1  0  0  1  1  0  0  1  0  1  0  0  1
[6,]  0  1  1  0  0  0  0  0  0  0  1  2  1  1  0  0  0  0  0  1  1  1  1  1  1
     79 80 81 82 83 84 85 86 87 88 89 90 91 92 93 94 95 96 97 98 99 100
[1,]  1  1  0  1  1  0  1  1  0  0  0  1  0  1  1  0  0  0  1  0  1   1
[2,]  1  1  2  0  0  0  1  2  0  2  1  0  0  2  1  0  0  0  0  1  1   2
[3,]  1  1  0  0  0  0  2  0  0  2  0  0  0  0  1  1  0  0  1  2  0   0
[4,]  0  1  2  0  1  0  1  2  0  0  0  0  0  0  1  0  0  0  1  0  0   0
[5,]  0  2  0  1  0  0  2  0  0  0  0  1  0  0  1  0  0  1  1  2  1   1
[6,]  2  1  0  0  1  0  0  0  0  2  0  1  0  0  0  0  0  1  1  0  1   1
...

Covariates:
data frame with 0 columns and 0 rows


Results:

$Qstatistic
         Q1       Q2       Q3
1 0.1910558 71.34471 16.59228

$p.value
       p.Q1         p.Q2       p.Q3
1 0.6620394 1.293491e-05 0.00011429
\end{Soutput}
\end{Schunk}

\subsection{Preprocessing with Partial Least Square Discriminant Analysis (PLS-DA)}

This method of data preprocessing is used for dichotomous outcome.

\begin{Schunk}
\begin{Sinput}
> library(rqt)
> data <- data.matrix(read.table(system.file("extdata/test.bin1.dat",
+                                            package="rqt"), header=TRUE))
> pheno <- data[,1]
> geno <- data[, 2:dim(data)[2]]
> colnames(geno) <- paste(seq(1, dim(geno)[2]))
> geno.obj <- SummarizedExperiment(geno)
> obj <- rqtClass(phenotype=pheno, genotype=geno.obj)
> res <- rQTest(obj, method="pls", out.type = "D", scale = TRUE)
\end{Sinput}
\begin{Soutput}
PLS-DA
200 samples x 100 variables and 1 response
standard scaling of predictors and response(s)
      R2X(cum) R2Y(cum) Q2(cum) RMSEE pre ort pR2Y  pQ2
Total   0.0187    0.301  -0.292   0.2   1   0  0.8 0.85
\end{Soutput}
\begin{Sinput}
> print(res)
\end{Sinput}
\begin{Soutput}
Phenotype:
[1] 1 1 1 1 1 1
...

Genotype:
     1 2 3 4 5 6 7 8 9 10 11 12 13 14 15 16 17 18 19 20 21 22 23 24 25 26 27 28
[1,] 0 0 0 1 0 2 1 0 0  0  2  1  2  0  0  2  1  1  0  0  1  0  0  0  0  2  1  0
[2,] 1 0 1 0 0 1 0 0 0  0  0  0  1  0  0  0  2  1  2  1  0  1  0  0  0  2  1  1
[3,] 0 0 0 0 1 0 0 1 0  1  1  1  0  0  0  1  1  0  0  1  1  1  0  0  0  1  0  0
[4,] 0 0 1 0 1 0 0 1 1  0  0  0  1  0  0  0  1  0  0  2  0  1  0  0  0  1  0  2
[5,] 0 0 1 1 1 1 1 1 0  1  0  1  0  0  0  0  0  0  0  0  1  0  0  1  0  1  2  0
[6,] 0 0 1 1 1 0 0 1 0  1  1  0  1  0  0  2  0  0  1  0  1  1  0  0  0  2  0  0
     29 30 31 32 33 34 35 36 37 38 39 40 41 42 43 44 45 46 47 48 49 50 51 52 53
[1,]  1  0  0  0  0  0  0  1  1  0  0  1  0  0  1  0  0  0  0  1  1  0  0  0  1
[2,]  2  0  0  0  1  2  1  2  0  1  1  0  1  0  0  1  0  0  0  2  0  1  0  1  0
[3,]  0  0  0  0  0  1  1  1  2  2  0  0  0  1  0  1  0  2  1  1  1  0  0  0  1
[4,]  0  0  0  1  2  2  0  1  1  1  1  0  0  0  1  1  0  0  1  1  2  0  1  0  2
[5,]  0  0  0  0  1  1  0  0  1  1  0  0  2  2  0  0  0  1  2  1  0  0  0  0  1
[6,]  0  0  0  0  1  0  1  1  0  0  2  0  1  1  2  0  0  0  1  1  1  0  0  1  1
     54 55 56 57 58 59 60 61 62 63 64 65 66 67 68 69 70 71 72 73 74 75 76 77 78
[1,]  1  1  1  1  0  0  0  0  1  1  0  1  1  1  0  1  1  2  1  1  1  0  0  2  1
[2,]  2  1  0  1  0  0  0  1  0  1  2  2  1  1  0  1  0  1  0  0  0  0  1  1  2
[3,]  2  1  0  0  0  0  0  0  0  1  0  0  1  1  0  1  2  1  0  1  1  0  0  0  0
[4,]  1  1  0  0  0  0  2  1  0  1  0  0  2  1  0  0  1  1  1  0  0  0  0  0  1
[5,]  1  1  1  0  0  0  2  1  1  0  0  1  1  0  0  1  1  0  0  1  0  1  0  0  1
[6,]  0  1  1  0  0  0  0  0  0  0  1  2  1  1  0  0  0  0  0  1  1  1  1  1  1
     79 80 81 82 83 84 85 86 87 88 89 90 91 92 93 94 95 96 97 98 99 100
[1,]  1  1  0  1  1  0  1  1  0  0  0  1  0  1  1  0  0  0  1  0  1   1
[2,]  1  1  2  0  0  0  1  2  0  2  1  0  0  2  1  0  0  0  0  1  1   2
[3,]  1  1  0  0  0  0  2  0  0  2  0  0  0  0  1  1  0  0  1  2  0   0
[4,]  0  1  2  0  1  0  1  2  0  0  0  0  0  0  1  0  0  0  1  0  0   0
[5,]  0  2  0  1  0  0  2  0  0  0  0  1  0  0  1  0  0  1  1  2  1   1
[6,]  2  1  0  0  1  0  0  0  0  2  0  1  0  0  0  0  0  1  1  0  1   1
...

Covariates:
data frame with 0 columns and 0 rows


Results:

$Qstatistic
        Q1       Q2       Q3
1 11.12606 2.322913 11.12606

$p.value
          p.Q1         p.Q2         p.Q3
1 0.0008512339 0.0008512339 0.0008512339
\end{Soutput}
\end{Schunk}

\subsection{Using additional covariates}

Quite often, researchers want to supply not only genetic 
data but also specific covariates, 
representic some physiological parameters or environment 
(for example, to evaluate 
hyphoteses of gene-environment interactions). 
In such cases, the package \texttt{rqt} 
can accept additional covariates, in form of 
\texttt{N} by \texttt{K} matrix, as provided below:

\begin{Schunk}
\begin{Sinput}
> library(rqt)
> data <- data.matrix(read.table(system.file("extdata/test.bin1.dat",
+                                            package="rqt"), header = TRUE))
> pheno <- data[,1]
> geno <- data[, 2:dim(data)[2]]
> colnames(geno) <- paste(seq(1, dim(geno)[2]))
> geno.obj <- SummarizedExperiment(geno)
> covars <- read.table(system.file("extdata/test.cova1.dat",package="rqt"), 
+     header=TRUE)
> obj <- rqtClass(phenotype=pheno, genotype=geno.obj, covariates = covars)
> res <- rQTest(obj, method="pca", out.type = "D")
> print(res)
\end{Sinput}
\begin{Soutput}
Phenotype:
[1] 1 1 1 1 1 1
...

Genotype:
     1 2 3 4 5 6 7 8 9 10 11 12 13 14 15 16 17 18 19 20 21 22 23 24 25 26 27 28
[1,] 0 0 0 1 0 2 1 0 0  0  2  1  2  0  0  2  1  1  0  0  1  0  0  0  0  2  1  0
[2,] 1 0 1 0 0 1 0 0 0  0  0  0  1  0  0  0  2  1  2  1  0  1  0  0  0  2  1  1
[3,] 0 0 0 0 1 0 0 1 0  1  1  1  0  0  0  1  1  0  0  1  1  1  0  0  0  1  0  0
[4,] 0 0 1 0 1 0 0 1 1  0  0  0  1  0  0  0  1  0  0  2  0  1  0  0  0  1  0  2
[5,] 0 0 1 1 1 1 1 1 0  1  0  1  0  0  0  0  0  0  0  0  1  0  0  1  0  1  2  0
[6,] 0 0 1 1 1 0 0 1 0  1  1  0  1  0  0  2  0  0  1  0  1  1  0  0  0  2  0  0
     29 30 31 32 33 34 35 36 37 38 39 40 41 42 43 44 45 46 47 48 49 50 51 52 53
[1,]  1  0  0  0  0  0  0  1  1  0  0  1  0  0  1  0  0  0  0  1  1  0  0  0  1
[2,]  2  0  0  0  1  2  1  2  0  1  1  0  1  0  0  1  0  0  0  2  0  1  0  1  0
[3,]  0  0  0  0  0  1  1  1  2  2  0  0  0  1  0  1  0  2  1  1  1  0  0  0  1
[4,]  0  0  0  1  2  2  0  1  1  1  1  0  0  0  1  1  0  0  1  1  2  0  1  0  2
[5,]  0  0  0  0  1  1  0  0  1  1  0  0  2  2  0  0  0  1  2  1  0  0  0  0  1
[6,]  0  0  0  0  1  0  1  1  0  0  2  0  1  1  2  0  0  0  1  1  1  0  0  1  1
     54 55 56 57 58 59 60 61 62 63 64 65 66 67 68 69 70 71 72 73 74 75 76 77 78
[1,]  1  1  1  1  0  0  0  0  1  1  0  1  1  1  0  1  1  2  1  1  1  0  0  2  1
[2,]  2  1  0  1  0  0  0  1  0  1  2  2  1  1  0  1  0  1  0  0  0  0  1  1  2
[3,]  2  1  0  0  0  0  0  0  0  1  0  0  1  1  0  1  2  1  0  1  1  0  0  0  0
[4,]  1  1  0  0  0  0  2  1  0  1  0  0  2  1  0  0  1  1  1  0  0  0  0  0  1
[5,]  1  1  1  0  0  0  2  1  1  0  0  1  1  0  0  1  1  0  0  1  0  1  0  0  1
[6,]  0  1  1  0  0  0  0  0  0  0  1  2  1  1  0  0  0  0  0  1  1  1  1  1  1
     79 80 81 82 83 84 85 86 87 88 89 90 91 92 93 94 95 96 97 98 99 100
[1,]  1  1  0  1  1  0  1  1  0  0  0  1  0  1  1  0  0  0  1  0  1   1
[2,]  1  1  2  0  0  0  1  2  0  2  1  0  0  2  1  0  0  0  0  1  1   2
[3,]  1  1  0  0  0  0  2  0  0  2  0  0  0  0  1  1  0  0  1  2  0   0
[4,]  0  1  2  0  1  0  1  2  0  0  0  0  0  0  1  0  0  0  1  0  0   0
[5,]  0  2  0  1  0  0  2  0  0  0  0  1  0  0  1  0  0  1  1  2  1   1
[6,]  2  1  0  0  1  0  0  0  0  2  0  1  0  0  0  0  0  1  1  0  1   1
...

Covariates:
          COV1
1 -0.612463927
2 -0.464158885
3  0.006153597
4 -0.732109468
5 -0.223530136
6 -0.744903822


Results:

$Qstatistic
        Q1       Q2       Q3
1 2.012625 3.761859 2.012625

$p.value
       p.Q1      p.Q2      p.Q3
1 0.1559952 0.8258796 0.2895166
\end{Soutput}
\end{Schunk}


\section{Meta-analysis}

\begin{Schunk}
\begin{Sinput}
> library(rqt)
> data1 <- data.matrix(read.table(system.file("extdata/phengen2.dat",
+                                             package="rqt"), skip=1))
> pheno <- data1[,1]
> geno <- data1[, 2:dim(data1)[2]]
> colnames(geno) <- paste(seq(1, dim(geno)[2]))
> geno.obj <- SummarizedExperiment(geno)
> obj1 <- rqtClass(phenotype=pheno, genotype=geno.obj)
> data2 <- data.matrix(read.table(system.file("extdata/phengen3.dat",
+                                             package="rqt"), skip=1))
> pheno <- data2[,1]
> geno <- data2[, 2:dim(data2)[2]]
> colnames(geno) <- paste(seq(1, dim(geno)[2]))
> geno.obj <- SummarizedExperiment(geno)
> obj2 <- rqtClass(phenotype=pheno, genotype=geno.obj)
> data3 <- data.matrix(read.table(system.file("extdata/phengen.dat",
+                                             package="rqt"), skip=1))
> pheno <- data3[,1]
> geno <- data3[, 2:dim(data3)[2]]
> colnames(geno) <- paste(seq(1, dim(geno)[2]))
> geno.obj <- SummarizedExperiment(geno)
> obj3 <- rqtClass(phenotype=pheno, genotype=geno.obj)
> res.meta <- rQTestMeta(list(obj1, obj2, obj3))
> print(res.meta)
\end{Sinput}
\begin{Soutput}
$final.pvalue
[1] 0.01995627

$pvalueList
[1] 0.006016377 0.367634896 0.245240026

$df
[1] 6

$chi.comb
[1] 15.03891
\end{Soutput}
\end{Schunk}


\section{Session information}
\begin{Schunk}
\begin{Sinput}
> sessionInfo()
\end{Sinput}
\begin{Soutput}
R version 3.3.2 (2016-10-31)
Platform: x86_64-apple-darwin13.4.0 (64-bit)
Running under: macOS Sierra 10.12.1

locale:
[1] C/UTF-8/C/C/C/C

attached base packages:
[1] parallel  stats4    stats     graphics  grDevices utils     datasets 
[8] methods   base     

other attached packages:
 [1] rqt_0.99.10                SummarizedExperiment_1.4.0
 [3] Biobase_2.34.0             GenomicRanges_1.26.1      
 [5] GenomeInfoDb_1.10.1        IRanges_2.8.1             
 [7] S4Vectors_0.12.1           BiocGenerics_0.20.0       
 [9] ropls_1.6.2                plyr_1.8.4                
[11] pls_2.5-0                  glmnet_2.0-5              
[13] foreach_1.4.3              Matrix_1.2-7.1            

loaded via a namespace (and not attached):
 [1] Rcpp_0.12.8        lattice_0.20-34    codetools_0.2-15   grid_3.3.2        
 [5] zlibbioc_1.20.0    CCP_1.1            XVector_0.14.0     iterators_1.0.8   
 [9] tools_3.3.2        CompQuadForm_1.4.2
\end{Soutput}
\end{Schunk}

\end{document}
